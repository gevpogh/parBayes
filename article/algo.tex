
{We apply the scheme of BA for probabilistic estimation of
the EoS given by vector parameter $\vec{\pi}$. Varying the three
parameters in the intervals $400<\varepsilon[\mathrm{MeV/fm^{3}}]<1200$,
$0<\gamma<1.5$ and $0.3<{c}_{s}^{2}<1.0$ with $N_{1}=N_{2}=N_{3}=13$
we explore a set of $N=2197$ hybrid EoS for each choice of hadronic
EoS. \cite{Blaschke:2013rma,Blaschke:2013ana,Benic:2014jia}).}

{Note that the occurrence of high-mass twins is testable by
observations. It requires precise radius measurements of 2M$_{\odot}$
pulsars in order to verify the existence of an almost horizontal branch
of high-mass pulsars with almost the same mass but radii differing
by up to a few kilometers. As a necessary condition for such sequences
the hadronic star sequence should have radii exceeding $12$ km. The
observation of this striking high-mass twin phenomenon would indicate
a strong first-order phase transition in neutron star matter which
gives important constraints for searches of a critical endpoint in
the QCD phase diagram.}

Fig.~\ref{flow:bayeser}...

(see the Algorithm \ref{algomaster})

{\small
\begin{algorithm}[auto]
\label{algomaster}
 \caption{Algorithm for initialization, domain decomposition and bookkeeping}
	\SetAlgoLined
	\KwData{{\tt MASTER} rank}
	\KwResult{Preparation of parallel simulations}
	Read and sort the band table\;
	Initializing the bookkeeping\;
	Define the domains as groups\;
	\While{infinitely}
	{
	Check and count free slaves\;
	\If{not enough free slaves}{go into waiting state}
	Check bookkeeping for bands to run\;
	\eIf{all groups fit}
		{
		\For{each group}
			{
			Send group size and band value to slaves\;
			Create new communicators\;
			Set the slaves as busy\;
			Set the groups as submitted\;
			}
		}
		{\eIf{biggest group does not fit}
			{Stop simulation\;}
			{\eIf{{\tt SLAVE}s are busy}
				{
				Go into waiting state\;
				Finished message received\;
				Set slaves free\;
				}
%				\tcc{Queue empty and all Slaves are free}
				{Finish simulation}
			}
		}
    }
\end{algorithm}
}