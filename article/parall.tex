-speedup whenn parallel

-scalability: up to which parallelit�t is code working

-load balance: small computations(e.g. small nc) in compare to big computations

Flow chat of regions to be parallelized.
Adding automated part to find extemum instead of visual manual analysis
% =================================================
% Set up a new layer for the debugging marks, and make sure it is on top
\pgfdeclarelayer{marx}
\pgfsetlayers{main,marx}
% A macro for marking coordinates (specific to the coordinate naming
% scheme used here). Swap the following 2 definitions to deactivate
% marks.
\providecommand{\cmark}[2][]{%
  \begin{pgfonlayer}{marx}
    \node [nmark] at (c#2#1) {#2};
  \end{pgfonlayer}{marx}
  }
\providecommand{\cmark}[2][]{\relax}
% -------------------------------------------------
% Start the picture
\begin{figure*}
	\centering
	% Define block styles
	\tikzstyle{decision} = [diamond, draw, fill=blue!20,text width=4.5em, text badly centered, node distance=3cm, inner sep=0pt]
	\tikzstyle{block} = [rectangle, draw, fill=blue!20,text width=5em, text centered, rounded corners, minimum height=4em]
	\tikzstyle{line} = [draw, -latex']
	\tikzstyle{cloud} = [draw, ellipse,fill=red!20, node distance=3cm,minimum height=2em]
	\tikzstyle{cir} = [circle, draw, densely dashed, fill=blue!20]
	\tikzset{
		aline/.style = {->, draw, ultra thick, >=triangle 60}
		base/.style={draw, on grid, align=center, minimum height=3em, text width=16em , minimum height=2em, node distance=6em, ultra thick},
		brick/.style = {draw, fill=red!20},
		dis/.style = {diamond, fill=green!20, text width=7em, aspect=3, node distance=8em},
		elli/.style = {ellipse, densely dashed, fill=red!20},
		rblock/.style = {brick, fill=blue!20, rounded corners},
	}
	\resizebox {0.95\textwidth} {!} {
	\begin{tikzpicture}[node distance = 2cm, auto]
		% Place nodes
		\node [block] (init) {initialize model};
		\node [cloud, left of=init] (expert) {expert};
		\node [cloud, right of=init] (system) {system};
		\node [brick, below of=init] (identify) {identify candidate models};
		\node [block, below of=identify] (evaluate) {evaluate candidate models};
		\node [elli, left of=evaluate, node distance=3cm] (update) {update model};
		\node [decision, below of=evaluate] (decide) {is best candidate better?};
		\node [cir, below of=decide, node distance=3cm] (stop) {stop};
	   % Draw edges
		\path [line] (init) -- (identify);
		\path [line] (identify) -- (evaluate);
		\path [line] (evaluate) -- (decide);
		\path [line] (decide) -| node [near start] {yes} (update);
		\path [line] (update) |- (identify);
		\path [line] (decide) -- node {no}(stop);
		\path [line,dashed] (expert) -- (init);
		\path [line,dashed] (system) -- (init);
		\path [line,dashed] (system) |- (evaluate);
	\end{tikzpicture}
	}

\caption{Flow diagram of the \bayeser for running \tover at 
different frequency bands in separate MPI worlds/groups. Bricks
represent usual algorithmic steps; diamonds are steps with decisions
about future direction of run; ellipses are steps when message passing
take place represented as dash-dotted paths; blocks with rounded corners
represent steps in which {\tt MASTER} or {\tt SLAVE} ranks are waiting
until a new MPI message is received.}
        \label{flow:bayeser}
\end{figure*}
