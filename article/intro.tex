{One of the question of the compact stars phsics is the discussions
of deconfinement transition in the cores of stars and whether it proceeds
as a crossover or rather as a first order transition. This question
is relevant for the problem of existence of a critical endpoint in
the QCD phase diagram, which is actual in present and upcoming heavy-ion
collision experiments.}

{The most basic features of a neutron star (NS) are the radius
$R$ and the mass $M$ which so far have not been well determined
simultaneously for a single object. In some cases masses are precisely
measured like in the case of binary systems, but radii are quite uncertain
\cite{Miller:2013tca}. In the other hand, for isolated neutron stars
some radius and mass measurements exist but lack the necessary precision
to allow conclusions about their interiors. In fact, it has been conjectured
that there exists a unique relation between $M$ and $R$ for all
neutron stars and their equation of state (EoS), thus constraining
the properties of their interior \cite{Lindblom1984}. For this reason,
accurate observations of masses and radii are crucial to study cold
dense nuclear matter expected to exist in neutron stars.}

{However, the presently observational data allow to make only
a probabilistic estimations of the internal structure of the star,
via Bayesian analysis (BA). This technique has been applied for the
first time to this problem by Steiner et al. \cite{Steiner:2010fz}
who exemplified very well the power of this method. In their analysis,
however, only a particular type of objects (X-ray bursters) has been
considered under strongly model dependent assumptions. In this work
is a continuation of our preliminary probabilistic studies of the
superdense stellar matter equation of state using Bayesian Analysis
and modeling of relativistic configurations of neutron stars \cite{Alvarez-Castillo:2014xea,Blaschke:2014via}
We put special emphasis on the choice of observational constraints
and focus on investigations of the possible existence of deconfined
quark matter in massive neutron stars such as the recently observed
$2~M_{\odot}$ pulsars \cite{Demorest:2010bx,Antoniadis:2013pzd}.}

The steps to archive the goals of modeling and implementing of the BA
are the following


\subsection{Making a the Star configuration from given EoS}

{The microscopical properties of compact stars are modeled
in the framework of general relativity, where the Einstein equations
are solved for a static (non-rotating), spherical star resulting in
the Tolman--Oppenheimer--Volkoff (TOV) equations \cite{Tolman:1939jz,Oppenheimer:1939ne,Glendenning:1997wn}
\begin{eqnarray}
\frac{dm(r)}{dr} & = & 4\pi r^{2}\varepsilon(r)\\
\frac{dp(r)}{dr} & = & -G\frac{(\varepsilon(r)+p(r))(m(r)+4\pi p(r)r^{3})}{r(r-2Gm(r))}\label{TOV1}
\end{eqnarray}
as well as the equation for the baryon number profile
\begin{equation}
\frac{dn_{B}(r)}{dr}=4\pi r^{2}m_{N}\frac{n_{B}(r)}{\sqrt{1-2Gm(r)/r}}~.
\end{equation}
These equations are integrated from the center of the star towards
its surface, with the radius of the star $R$ being defined by the
condition $p(R)=0$ and the gravitational mass by $M=m(R)$. In a
similar manner, the baryon mass of the star is given by $M_{B}=m_{N}n_{B}(R)$,
where $m_{N}$ is the nucleon mass.}

{To complete the solution to the TOV equations the EoS is required.
It is given by the relation $p=p(\varepsilon)$ which carries information
about the microscopic ingredients of the dense nuclear matter, as
mentioned before. Thus, the above equations have to be solved simultaneously
using the equation of state under boundary conditions at the star
centre ($r=0$), taken as an input. In this way, for a given value
of $\varepsilon(r=0)$ the solution of the TOV equations are the $p(r)$
and $m(r)$ profiles and with them the parametric relationship $M(R)$
can be obtained.}

{%\section{The chosen EoS}
For the present study we use the scheme suggested by Alford, Han and
Prakash~\cite{Alford:2013aca} for defining the hybrid EoS (shorthand:
AHP scheme),
\begin{equation}
p(\varepsilon)=p_{h}(\varepsilon)\Theta(\varepsilon_{H}-\varepsilon)+p_{h}(\varepsilon_{H})\Theta(\varepsilon-\varepsilon_{H})\Theta(\varepsilon_{H}+\Delta\varepsilon-\varepsilon)+p_{q}(\varepsilon)\Theta(\epsilon-\epsilon_{H}-\Delta\epsilon),
\end{equation}
where $p_{h}(\varepsilon)$ is a hadronic matter EoS and $p_{q}(\varepsilon)$
represents the high density matter phase, here considered as deconfined
quark matter with the bag model type EoS
\begin{equation}
p_{q}(\varepsilon)=c_{q}^{2}\varepsilon-B.
\end{equation}
The stiffness of this EoS is given by $c_{q}^{2}$, the squared speed
of sound. The positive bag constant $B$ assures confinement, i.e.
the dominance of the hadronic EoS at low densities. Note that a parametrization
in this form by Haensel et al.~\cite{Zdunik:2012dj} describes pretty
well the superconducting NJL model derived in~\cite{Klahn:2006iw}
and systematically explored in~\cite{Klahn:2013kga}. In the AHP
scheme, the hadronic EoS is fixed and not subject to parametric variations.
However, we shall use different well-known model EoS in our study
the nonrelativistic variational EoS of Akmal et al. \cite{Akmal:1998cf}
(APR) and the density-dependent relativistic meanfield EoS of Typel
and Wolter \cite{Typel:1999yq} (DD2) in its parametrization from
Ref.~\cite{Typel:2009sy}. Both EoS come eventually with extensions
due to an excluded volume for baryons, as described for DD2 in \cite{Benic:2014jia}.
}


{The free parameters of the model are the transition density
$\varepsilon_{H}$, the energy density jump $\Delta\varepsilon\equiv\gamma\varepsilon_{H}$
and $c_{q}^{2}$. The set of hybrid EoS in the plane pressure versus
energy density is shown in Fig.~\ref{APH_Scheme} for APR as the
hadronic EoS. %without (left panel) and with (right panel) baryonic excluded volume for the variation
%of the set of input parameters as described in the next section.
}


\subsection{BA Formulation and Formalization}

{We define the vector of free parameters $\overrightarrow{\pi}\left(\varepsilon_{c},\gamma,{c}_{q}^{2}\right)$
%, where $\varepsilon_H$ is the critical value of energy density at the onset of the phase
%transition (PhT), $\gamma = \Delta epsilon / \epsilon_H$
%is a ratio of the energy density jump at the PhT to the critical one, and ${c}_{q}^{2}$
%is square of speed of sound in quark matter.
defining the hybrid EoS with a first order phase transition from nuclear
to quark matter. %The nuclear equation of state can be taken~APR~\cite{Akmal1998}.
}

{These parameters are sampled
\begin{equation}
\pi_{i}=\overrightarrow{\pi}\left(\varepsilon_{k},\gamma_{l},{{c}_{q}^{2}}_{m}\right),\label{pi_vec}
\end{equation}
where $i=0\dots N-1$ (here $N=N_{1}\times N_{2}\times N_{3}$) as
$i=N_{1}\times N_{2}\times k+N_{2}\times l+m$ and $k=0\dots N_{1}-1$,
$l=0\dots N_{2}-1$, $m=0\dots N_{3}-1$, here $N_{1}$, $N_{2}$
and $N_{3}$ denote the number of parameters for $\varepsilon_{k}$,
$\gamma_{l}$ and ${{c}_{q}^{2}}_{m}$, respectively. Solving the
TOV equations with varying boundary conditions for $\varepsilon(r=0)$
generates a sequence of $M(R)$ curves characteristic for each of
these $N$ EoS. Subsequently, different neutron star observations
with their error margins can be used to assign a probability to each
choice in from the set of EoS parameters. We use three constraints:
(i) the mass constraint for PSR J0348+0432~\cite{Antoniadis:2013pzd},
(ii) the radius constraint for PSR J0437-4715~\cite{Bogdanov:2012md}
and (iii) the constraint on the baryon mass at the well measured gravitational
mass for the star B in the double pulsar system PSR J0737-3039~\cite{Kitaura:2006bt},
which improved the earlier suggestion by Podsiadlowski et al.~\cite{Podsiadlowski:2005ig}.}

{The goal is to find the set of most probable $\pi_{i}$ based
on given constraints using Bayesian Analysis (BA). For initializing
BA we propose that }{a priori}{ each vector of parameter
$\pi_{i}$ has probability equal one: $P\left(\pi_{i}\right)=1$ for
$\forall i$.}


\subsection{Inputs from Observations}


\subsubsection{Mass constraint for PSR~J0348+0432}

{We propose that the error of this measurement is normal distributed
$\mathcal{N}(\mu_{A},\sigma_{A}^{2})$, where $\mu_{A}=2.01~\mathrm{M_{\odot}}$
and $\sigma_{A}=0.04~\mathrm{M_{\odot}}$ are measured for PSR~J0348+0432~\cite{Antoniadis:2013pzd}.
Using this assumption we can calculate conditional probability of
the event $E_{A}$ that the mass of the neutron star corresponds to
this measurement
\begin{equation}
P\left(E_{A}\left|\pi_{i}\right.\right)=\Phi(M_{i},\mu_{A},\sigma_{A}),\label{p_anton}
\end{equation}
where $M_{i}$ is the maximum mass obtained for $\pi_{i}$ and $\Phi(x,\mu,\sigma)$
is the cumulative distribution function for the normal distribution.
%\begin{equation}
%\label{Laplas}
%\Phi(x, \mu, \sigma) = \frac{1}{2}\left[1+\erf\left(\frac{x-\mu}{\sqrt{2\sigma^2}}\right)\right].
%\end{equation}
}


\subsubsection{Radius constraint for PSR~J0437-4715}

{Recently, a radius constraint for the nearest millisecond
pulsar PSR~J0437-4715 have been obtained~\cite{Bogdanov:2012md}
giving $\mu_{B}=15.5~\mathrm{km}$ and $\sigma_{B}=1.5~\mathrm{km}$.
With these data one calculates the conditional probability of the
event $E_{B}$ that the radius of a neutron star corresponds to this
measurement
\begin{equation}
P\left(E_{B}\left|\pi_{i}\right.\right)=\Phi(R_{i},\mu_{B},\sigma_{B}).\label{p_bogdan}
\end{equation}
%A similar constraint has been reported for RX J1856.5-3754 by Hambaryan et
%al.~\cite{Hambaryan:2013}.
}


\subsubsection{$M$--$M_{B}$ Relation Constraint for PSR J0737-3039(B)}

{This constraint gives a region in the $M$--$M_{B}$ plane.
We need to estimate the probability of a point ${\cal M}_{i}=\left({M}_{i},{M_{B}}_{i}\right)$
to be close to the point $\mu=\left(\mu_{G},\mu_{B}\right)$. The
mean values $\mu_{G}=1.249$, $\mu_{B}=1.36$ and standard deviations
$\sigma_{M}=0.001$, $\sigma_{M_{B}}=0.002$ are given in \cite{Kitaura:2006bt}.
The probability can be calculated by following formula:
\begin{equation}
P\left(E_{K}\left|\pi_{i}\right.\right)=\left[\Phi\left(\xi_{G}\right)-\Phi\left(-\xi_{G}\right)\right]\cdot\left[\Phi\left(\xi_{B}\right)-\Phi\left(-\xi_{B}\right)\right],\label{p_kitaura}
\end{equation}
where $\Phi\left(x\right)=\Phi\left(x,0,1\right)$, $\xi_{G}={\displaystyle \frac{\sigma_{M}}{d_{M}}}$
and $\xi_{B}={\displaystyle \frac{\sigma_{M_{B}}}{d_{M_{B}}}}$, $d_{M}$
and $d_{M_{B}}$ are absolute values of components of vector $\mathrm{{d}}=\mathrm{{\bf \mu}}-\mathrm{{M}}_{i}$,
here $\mathrm{{\bf \mu}}=\left(\mu_{G},\mu_{B}\right)^{T}$ was given
in \cite{Kitaura:2006bt} and $\mathrm{{M}}_{i}=\left({M}_{i},{M_{B}}_{i}\right)^{T}$
stems for the solution of TOV equations for the $i^{\mathrm{th}}$
vector of EoS parameters $\pi_{i}$. Note that formula (\ref{p_kitaura})
does not correspond to a multivariate normal distribution.}


\subsection{Calculation of {a posteriori} Probabilities}

{Note, that these measurements are independent on each other.
Therefore, the complete conditional probability of the event that
a compact object constructed with an EoS characterized by $\pi_{i}$
fulfils all constraints is
\begin{equation}
P\left(E\left|\pi_{i}\right.\right)=P\left(E_{A}\left|\pi_{i}\right.\right)\times P\left(E_{B}\left|\pi_{i}\right.\right)\times P\left(E_{K}\left|\pi_{i}\right.\right).\label{p_event}
\end{equation}
Now, we can calculate probability of $\pi_{i}$ using Bayes' theorem:
\begin{equation}
P\left(\pi_{i}\left|E\right.\right)=\frac{P\left(E\left|\pi_{i}\right.\right)P\left(\pi_{i}\right)}{\sum\limits _{j=0}^{N-1}P\left(E\left|\pi_{j}\right.\right)P\left(\pi_{j}\right)}.\label{pi_apost}
\end{equation}
}
